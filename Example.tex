\documentclass{LoLaTeXcv}
\usepackage[T1]{fontenc}
\usepackage[utf8]{inputenc}
\usepackage[french]{babel}

\begin{document}
	
\lltxPersonalInfo{
	Antoine Hugounet}{
	15, avenue Pasteur \\
	95400 Arnouville \\
	22 ans}{
	Étudiant, Master 1 de mathématiques \\
	Sorbonne Université}{
	06 52 09 81 70 \\
	antoine@hugounet.com \\
	\url{https://github.com/kryzar}}

\lltxTitle{Apprentissage}

\begin{lltxHistory}{Cursus}
	\item[2019-2020] M1 de mathématiques \lltxdotfill Sorbonne Université
	\item[2018-2019] L3 de mathématiques, mention bien \lltxdotfill Sorbonne Université
	\item[2017] Erasmus+ (premier semestre) \lltxdotfill Université d'Oslo (Norvège)
	\item[2015-2018] licence de mécanique, mention bien \lltxdotfill Sorbonne Université
	\item[2015-2018] licence de musicologie, mention assez bien \lltxdotfill Sorbonne Université
	\item[2015] Bac S, spécialité mathématiques, mention très bien \lltxdotfill Lycée Rocroy St-Vincent de Paul
\end{lltxHistory}

\begin{lltxHistory}{Stages}
	\item[2020] développement d'un assistant de preuve pédagogique basé sur \textit{Lean} \lltxdotfill Frédéric Le Roux (IMJ-PRG)
	\item[2019] initiation à la recherche, partage du quotidien des chercheurs \lltxdotfill Frédéric Le Roux (IMJ-PRG)
	\item[2013] initiation à la recherche, partage du quotidien des chercheurs \lltxdotfill Amal Attouchi (LAGA)
	\item[2012] assistance des journalistes \lltxdotfill Denis Cheissoux (Radio France)
\end{lltxHistory}

\begin{lltxItemize}{Lectures assidues}
	\item{\textit{Algèbre et théories galoisiennes}, R. et A. Douady : Théorème de Zorn, Catégories et foncteurs}
	\item{\textit{Algèbre : le grand combat}, G. Berhuy : théorie de Galois}
	\item{\textit{Carmichael ideals in number rings}, G. A. Steele, Journal of Number Theory : en entier}
	\item{\textit{Cours d'algèbre}, M. Demazure : Corps finis}
	\item{Challenges \url{https://www.mathraining.be} : niveau \textit{Compétent}}
	\item{\textit{Cryptography I}, D. Boneh, cours au MIT : Stream ciphers, Block ciphers}
	\item{\textit{Raisonnements divins}, M. Aigner, M. G. Ziegler : Théorie des nombres}
\end{lltxItemize}

\begin{lltxHistory}{Informatique}
\item[Programmation]C++, C, Python \& Numpy, Sage, Pari/GP, Octave/Matlab, HTML/CSS/JavaScript
	\item[Outils]Git, neovim, \LaTeX, Bash/zsh, Anki, Gnuplot, Pandoc, XCode, ffmpeg
	\item[Systèmes]GNU/Linux (bases, terminal), macOS (usage avancé), Windows (bases)
\end{lltxHistory}

\lltxTitle{Investissement}

\begin{lltxItemize}{Projets actuels}
	\item Développement d'un assistant de preuve pédagogique basé sur \textit{Lean}, avec Frédéric Le Roux (IMJ-PRG) et Florian Dupeyron (étudiant en Master 1 d'informatique, Sorbonne Université). Notre but est de créer un outil aidant les élèves à formaliser leur intuition et les encourrager à la rédaction. Nous écrivons pour cela un \textit{serveur logique} en Python, qui fait le lien entre une interface graphique et l'API de \textit{Lean}. Nous choisissons Python car ce langage rend très simple l'écriture de modules externes, notamment par les étudiants qui ne connaissent que ce langage.
	\item Création et organisation d'un séminaire de mathématiques dans les catacombes de Paris.
\end{lltxItemize}

\begin{lltxItemize}{Réalisations}
	\item programmes en C++ : équation de Laplace, problème à $n$ corps, autres algorithmes standards
	\item court texte intitulé \textit{Study on opera with the example of Oslo Opera's Wozzeck}
	\item analyse musicale de la pièce de musique spectrale \textit{Cloches d'adieu, et un de diffusion sourire} (T. Murail)
	\item critiques pour le service culturel de l'université Paris-Sorbonne
\end{lltxItemize}

\vspace{1em}
\textit{Les réalisations sont disponibles sur GitHub : \url{https://github.com/kryzar/Regolith.}}
\lltxTitle{Miscellannées}

\begin{lltxHistory}{Expériences professionnelles}
	\item[2016-présent] cours particuliers de mathématiques, lycée et collège 
	\item[2018] ouvrier paysagiste \lltxdotfill Val d'Oise Jardins
	\item[2017] vendeur de boissons au Stade de France \lltxdotfill Le Désoiffeur
\end{lltxHistory}

\begin{lltxHistory}{Passions}
	\item[Sport] plongée sous-marine et apnée, randonnée, musculation, cyclisme
	\item[Musique] violon (dans un ensemble baroque ou seul), basse, musicologie
	\item[Autres] jardinage, ktaphilie, programmation, minéralogie, photographie, road-trips, sport automobile
\end{lltxHistory}

\begin{lltxHistory}{Compétences pratiques}
	\item[Langues] anglais (C1), allemand (A1), italien (débutant)
	\item[Mobilité] permis B
\end{lltxHistory}


\end{document}
